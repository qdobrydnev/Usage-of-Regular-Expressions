\documentclass{article}
\usepackage{graphicx} % Required for inserting images
\usepackage{indentfirst} % Required for adding indent

\title{Usage of Regular Expressions in text processing}
\author{Dobrydnev Nikolai}
\date{October 2023}

\begin{document}

\maketitle

\section{Introduction}
\indent This article aims to provide information on the the utilization of \textit{Regular Expressions} when writing, processing and retrieving data. \par
In this article I am going to define the term \textit{Regular Expression}, explore the history of its development, examine algorithms it deploys and analyze its contribution to working with data. 

\section{Definition}
\indent Regular Expression (\textit{RegEx}) is a set of characters based on strict rules that represents particular pattern to find in the text. It is widely used in finding numerous variations of strings that follow the same pattern. \par

One example of Regular Expression can be: \verb|https?://[^\s]+| \par
While \textit{RegEx} instantly identifies all the valid links  (\textit{http} or \textit{https} protocols) in the text, \textit{common search} (which implies matching sub string) requires no less than two queries for the same outcome. \par

Regular Expression is a beneficial tool to process any kind of text or data. Apart from its convenience, it provides an opportunity to create complex queries that cannot be achieved through the common search.
\end{document}
